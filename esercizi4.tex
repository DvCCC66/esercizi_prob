\documentclass{article}
\usepackage{graphicx} % Required for inserting images
\usepackage{amsmath}
\usepackage{listings}
\usepackage{textcomp}
\usepackage{multirow}
\usepackage{multicol}
\usepackage{booktabs}
\usepackage{graphicx}
\usepackage{floatflt}
\usepackage{epsfig}
\usepackage{pstricks}
\usepackage{subfigure}
\usepackage[labelfont=bf, font=scriptsize]{caption}
\usepackage[italian]{varioref}
\usepackage[suftesi]{frontespizio}
\usepackage{color}
\usepackage{tikz}
\usepackage{caption}
\usepackage{pgfplots}
\usepackage{comment}
\usepackage{lipsum}
\usepackage{hyperref}
\pgfplotsset{compat=1.16}
\usepackage[table]{xcolor}% http://ctan.org/pkg/xcolor
\usepackage{phoenician}
\usepackage{dsfont}

\usepackage{subfigure}
\usepackage[export]{adjustbox} % per l'allineamento delle immagini
\usepackage{float}
\usepackage{amssymb} % for the \lesssim symbol
\usepackage{makecell}
\usepackage{changepage}
\usepackage{amssymb}
\newcommand{\myrightarrow}[1]{\xrightarrow{\makebox[2em][c]{$\scriptstyle#1$}}}
\newcommand{\convlegge}{\myrightarrow{\mathcal{L}}}

\newcommand{\ind}{\mathrel{\perp\!\!\!\perp}}
\title{Esercizi prob 4}
\author{Davide Caucchiolo}
\begin{document}
\maketitle
\section{Esercizio 4.1}
Sia $\widetilde{K}$ tale che $\mathbb{P}\left(X\in \widetilde{K}\right) \geq  1-\frac{\varepsilon }{2}$ (esiste perché... credo sia una proprietà degli integrali o roba simile...).

La convergenza in legge implica che:
\[
	\mathbb{P}\left(X_n\in \widetilde{K}\right) \myrightarrow{n} \mathbb{P}\left(X\in \widetilde{K}\right)
\]

Quindi
\[
	\exists N_0,\ \ \ \forall n\geq N_0,\ \ \ \ \mathbb{P}\left(X_n\in \widetilde{K}\right)\geq 1-\varepsilon 
\]

Per $n<N_0$ definiamo $K_n$ tale che:
\[
	\mathbb{P}\left(X\in K_n\right) \geq  1-\varepsilon 
\]

Quindi in $K:= \bigcup_{n<N_0} K_n \cup  \widetilde{K}$ (che è compatto perché unione finita di compatti):
\[
	\mathbb{P}\left(X\in K\right) \geq  1-\varepsilon\ \ \ \ \ \forall n
\]

Ovvero $\left(X_n\right)_{n\in \mathbb{N}}$ è tight.


\section{Esercizio 4.2}
\[
	y:= \sup_n \mathbb{E}\left[h\left(\left|X_n\right|\right)\right] < \infty 
\]
\[
	\mathbb{E}\left[h\left(\left|X_n\right|\right)\right] \leq  \mathbb{E}\left[y\right] = y\ \ \ \ \forall n
\]

Applichiamo ora la disuguaglianza di Markov:
\[
	\mathbb{P}\left(h\left(\left|X_n\right|\right) \geq  k\right) \leq  \frac{\mathbb{E}\left[h\left(\left|X_n\right|\right)\right]}{k} \leq  \frac{y}{k}\ \ \ \ \ \forall n
\]
\[
	\mathbb{P}\left(h\left(\left|X_n\right|\right) < k\right) = 1- \mathbb{P}\left(h\left(\left|X_n\right|\right) \geq  k\right) \geq  1-\frac{y}{k}\ \ \ \ \ \forall n
\]

Quindi per la tightness basta predere l'insieme compatto $\overline{\left\{h\left(\left|X_n\right|\right) < k\right\}}$ (la linea sopra indica la chiusura dell'insieme, che è già limitato siccome $h$ è crescente e tendente all'infinito).




\end{document}
