\documentclass{article}
\usepackage{graphicx} % Required for inserting images
\usepackage{amsmath}
\usepackage{listings}
\usepackage{textcomp}
\usepackage{multirow}
\usepackage{multicol}
\usepackage{booktabs}
\usepackage{graphicx}
\usepackage{floatflt}
\usepackage{epsfig}
\usepackage{pstricks}
\usepackage{subfigure}
\usepackage[labelfont=bf, font=scriptsize]{caption}
\usepackage[italian]{varioref}
\usepackage[suftesi]{frontespizio}
\usepackage{color}
\usepackage{tikz}
\usepackage{caption}
\usepackage{pgfplots}
\usepackage{comment}
\usepackage{lipsum}
\usepackage{hyperref}
\pgfplotsset{compat=1.16}
\usepackage[table]{xcolor}% http://ctan.org/pkg/xcolor
\usepackage{phoenician}
\usepackage{dsfont}

\usepackage{subfigure}
\usepackage[export]{adjustbox} % per l'allineamento delle immagini
\usepackage{float}
\usepackage{amssymb} % for the \lesssim symbol
\usepackage{makecell}
\usepackage{changepage}
\usepackage{amssymb}
\newcommand{\myrightarrow}[1]{\xrightarrow{\makebox[2em][c]{$\scriptstyle#1$}}}
\newcommand{\convlegge}{\myrightarrow{\mathcal{L}}}

\newcommand{\ind}{\mathrel{\perp\!\!\!\perp}}
\title{Esercizi prob 4}
\author{Davide Caucchiolo}
\begin{document}
\maketitle
\section{Esercizio 4.1}
Sia $\widetilde{K}$ tale che $\mathbb{P}\left(X\in \widetilde{K}\right) \geq  1-\frac{\varepsilon }{2}$ (esiste perché... credo sia una proprietà degli integrali o roba simile...).

La convergenza in legge implica che:
\[
	\mathbb{P}\left(X_n\in \widetilde{K}\right) \myrightarrow{n} \mathbb{P}\left(X\in \widetilde{K}\right)
\]

Quindi
\[
	\exists N_0,\ \ \ \forall n\geq N_0,\ \ \ \ \mathbb{P}\left(X_n\in \widetilde{K}\right)\geq 1-\varepsilon 
\]

Per $n<N_0$ definiamo $K_n$ tale che:
\[
	\mathbb{P}\left(X\in K_n\right) \geq  1-\varepsilon 
\]

Quindi in $K:= \bigcup_{n<N_0} K_n \cup  \widetilde{K}$ (che è compatto perché unione finita di compatti):
\[
	\mathbb{P}\left(X\in K\right) \geq  1-\varepsilon\ \ \ \ \ \forall n
\]

Ovvero $\left(X_n\right)_{n\in \mathbb{N}}$ è tight.


\section{Esercizio 4.2}
\[
	y:= \sup_n \mathbb{E}\left[h\left(\left|X_n\right|\right)\right] < \infty 
\]
\[
	\mathbb{E}\left[h\left(\left|X_n\right|\right)\right] \leq  \mathbb{E}\left[y\right] = y\ \ \ \ \forall n
\]

Applichiamo ora la disuguaglianza di Markov:
\[
	\mathbb{P}\left(h\left(\left|X_n\right|\right) \geq  k\right) \leq  \frac{\mathbb{E}\left[h\left(\left|X_n\right|\right)\right]}{k} \leq  \frac{y}{k}\ \ \ \ \ \forall n
\]
\[
	\mathbb{P}\left(h\left(\left|X_n\right|\right) < k\right) = 1- \mathbb{P}\left(h\left(\left|X_n\right|\right) \geq  k\right) \geq  1-\frac{y}{k}\ \ \ \ \ \forall n
\]

Quindi per la tightness basta predere l'insieme compatto $\overline{\left\{h\left(\left|X_n\right|\right) < k\right\}}$ (la linea sopra indica la chiusura dell'insieme, che è già limitato siccome $h$ è crescente e tendente all'infinito).


\section{Esercizio 4.3}
\subsection{(1): $\int_{-T}^{T} \frac{e^{-ita}-e^{-itb}}{it} \phi (t)\,dt = \int_{\mathbb{R}} \left(\int_{-T}^{T} \frac{e^{it(x-a)} - e^{it(x-b)}}{it} \, dt\right) P_X(dx)$}

\[
	\int_{-T}^{T} \frac{e^{-ita}-e^{-itb}}{it} \phi (t)\,dt = \int_{-T}^{T} \frac{e^{-ita}-e^{-itb}}{it} \left(\int_{\Omega } e^{it X(\omega )} \mathbb{P(d\omega )}\right) \,dt = \int_{-T}^{T} \frac{e^{-ita}-e^{-itb}}{it} \left(\int_{\mathbb{R}} e^{itx} P_X(dx)\right) \,dt 
\]
Nota: questo passaggio non so come giustificarlo bene. Per favore qualche matematico spieghi meglio cosa sta succedendo.

Ora vogliamo controllare se i due integrali sono (assolutamente) integrabili:
\[
	\int_{\mathbb{R}} \left|e^{itx} P_X(dx)\right| = \int_{\mathbb{R}} \left|P_X\left(dx\right)\right| < +\infty
\]
\[
	\int_{-T}^{T} \left|\frac{e^{-ita} - e^{-itb}}{it}\right|\,dt =\]\[
	\int_{-T}^{T} \left|\frac{1-ita-1+itb+O(t^2)}{t}\right|\,dt = \]\[
	\int_{-T}^{T} \left|\frac{it\left(b-a\right) + O(t^2)}{t}\right|\,dt =\]\[
	\int_{-T}^{T} \left|i(b-a) + O(t)\right|\,dt < +\infty 
\]

Quindi applicando Fubini+Tonelli e distribuendo si ottiene la tesi.

\subsection{(2): $\frac{1}{2\pi } \int_{-T}^{T}\frac{e^{it(x-a)} - e^{it(x-b)}}{it} \,dt = \frac{1}{\pi }\left[sgn(x-a) S\left(\left|x-a\right|T\right) - sgn(x-b)S\left(\left|x-b\right|T\right)\right]$}

\[
	\frac{1}{2\pi } \int_{-T}^{T}\frac{e^it(x-a)}{it} \,dt = \]\[
	\frac{1}{\pi } sgn(x-a) \int_{0}^{T} \left[\frac{e^{it\left|x-a\right|} - e^{-it\left|x-a\right|}}{2it}\right] \,dt =\]\[
	\frac{1}{\pi } sgn(x-a) \int_{0}^{T} \frac{\sin\left(\left|x-a\right|t\right)}{t} \,dt
\]

\subsection{(3): $\lim_{T\nearrow \infty }  \int_{-T}^{T} \frac{e^{-ita}-e^{-itb}}{it} \phi (t)\,dt = P_X((a,b)) + \frac{1}{2}P_X\left(\left\{a\right\}\right) + \frac{1}{2}P_X(\left\{b\right\})$}
\[
	\lim_{T\nearrow \infty }  \int_{-T}^{T} \frac{e^{-ita}-e^{-itb}}{it} \phi (t)\,dt = \]\[
	\lim_{T\nearrow \infty }  \frac{1}{\pi }\left[sgn(x-a) S\left(\left|x-a\right|T\right) - sgn(x-b)S\left(\left|x-b\right|T\right)\right] P_X(dx)=
\]
\[
	\frac{1}{\pi } \left[ \int_{x<a} \left(-\frac{\pi }{2} + \frac{\pi }{2} \right)P_X(dx) + \frac{\pi }{2}P_X\left(\left\{a\right\}\right) + \int_{a<x<b} \left(\frac{\pi }{2} + \frac{\pi }{2} \right)P_X(dx) + \frac{\pi }{2}P_X\left(\left\{b\right\}\right) + \int_{x>b} \left(\frac{\pi }{2} - \frac{\pi }{2} \right)P_X(dx)\right]
\]

\end{document}
