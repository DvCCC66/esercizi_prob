\documentclass{article}
\usepackage{graphicx} % Required for inserting images
\usepackage{amsmath}
\usepackage{listings}
\usepackage{textcomp}
\usepackage{multirow}
\usepackage{multicol}
\usepackage{booktabs}
\usepackage{graphicx}
\usepackage{floatflt}
\usepackage{epsfig}
\usepackage{pstricks}
\usepackage{subfigure}
\usepackage[labelfont=bf, font=scriptsize]{caption}
\usepackage[italian]{varioref}
\usepackage[suftesi]{frontespizio}
\usepackage{color}
\usepackage{tikz}
\usepackage{caption}
\usepackage{pgfplots}
\usepackage{comment}
\usepackage{lipsum}
\usepackage{hyperref}
\pgfplotsset{compat=1.16}
\usepackage[table]{xcolor}% http://ctan.org/pkg/xcolor
\usepackage{phoenician}
\usepackage{dsfont}

\usepackage{subfigure}
\usepackage[export]{adjustbox} % per l'allineamento delle immagini
\usepackage{float}
\usepackage{amssymb} % for the \lesssim symbol
\usepackage{makecell}
\usepackage{changepage}
\usepackage{amssymb}
\usepackage{geometry}
\newcommand{\myrightarrow}[1]{\xrightarrow{\makebox[2em][c]{$\scriptstyle#1$}}}
\newcommand{\convlegge}{\myrightarrow{\mathcal{L}}}
\newcommand{\re}{\mathfrak{Re}}
\newcommand{\im}{\mathfrak{Im}}
\geometry{margin=15mm}
\setlength{\columnseprule}{.5pt}
\setlength{\oddsidemargin}{30pt}
\setlength{\hoffset}{-1in}
%\setlength{\parindent}{-10pt}


\newcommand{\ind}{\mathrel{\perp\!\!\!\perp}}
\title{Esercizi prob 5}
\author{Davide Caucchiolo}
\begin{document}
\section{Esercizio 5.1}
\[\int_{\{|X|>1\}}\left|X\right| d\omega \leq \int_{|X|>1} \left|X\right|^p d\omega \leq c\]
Quindi, definendo $Y:=X \mathds{1}_{|X|>1}$, si ha che $Y\in \mathbb{L}^1$, quindi tesi (basta imporre che $L>1$).

\section{Esercizio 5.2}
\subsection{$\Rightarrow $}
Prima tesi:
\[
	\sup_{X\in \mathcal{C}} \|X\|_1 
	= \sup_{X\in \mathcal{C}} \left(\mathbb{E}\left[\left|X\right|; \left|X\right| > L\right] + \mathbb{E}\left[\left|X\right|; \left|X\right| \leq  L\right]\right) 
	\leq  \varepsilon +L < \infty     
\]    

Per la seconda tesi, sia $G = G_L \cup G_b$, con $G_L:= G\cap \left\{|X|>L\right\}$ (tale che $\mathbb{E}\left[\left|X\right|; \left|X\right| > L\right] < \frac{\varepsilon }{2}$) e $G_b := G \setminus G_L$. Scegliendo $\delta <\frac{\varepsilon }{2L}$ si ha:
\[
	\sup_{X\in \mathcal{C}} \left(\mathbb{E}\left[|X|;G_L\right] + \mathbb{E}\left[|X|;G_b\right]\right) \leq  
	\frac{\varepsilon }{2} + \frac{\varepsilon }{2} = \varepsilon 
\]

\subsection{$\Leftarrow $}
\[c:= \sup_{X\in \mathcal{C}} \|X\|_1\]

Dobbiamo dimostrare che $\forall \delta >0,\ \exists L,\ \forall X,\ \mathbb{P}\left(|X|>L\right)<\delta $:

\[\|X\|_1 \geq  L\mathbb{P}\left(|X|>L\right)\]
\[\mathbb{P}\left(|X|>L\right) \leq  \frac{\|X\|_1}{L} \leq  \frac{c}{L} < \delta \]
(con $L$ sufficientemente grande).


\section{Esercizio 5.3}
$X_n\in \mathbb{L}^1$: già dimostrato nell'esercizio precedente (prima tesi di parte "$\Rightarrow $").

Tesi $\lim_n\mathbb{E}\left[X_n\right] = \mathbb{E}\left[X\right]$:
Sappiamo già che, data la convergenza in legge, $\lim_n\mathbb{E}\left[h(X_n)\right] = \mathbb{E}\left[h\left(X\right)\right]\ \ \ \forall h\in \mathcal{C}^0_b$.

Questo vale anche per $h = \mathds{1}_{\{|X|>L\}}$ (ovviamente questo $h$ non è continuo, ma può essere approssimato da funzioni continue): $X$ è UI (quindi è anche $\mathbb{L}^1$).

Inoltre questo vale anche con $h\left(x\right) =x$ (anche questo non bounded ma approssimabile da bounded).


(Sinceramente a me questo esercizio sussa perché sembra già tutto fatto)


\section{Esercizio 5.4}
Usiamo l'esercizio precedente:

\[Y_n := f\left(\frac{S_n}{\sqrt{n}}\right)\]
\[Y:=f(Z)\]

\subsection{Tesi 1: $Y_n \convlegge Y$}
Tesi: $\mathbb{E}\left[h\left(Y_n\right)\right] \rightarrow  \mathbb{E}\left[h\left(Y\right)\right]\ \ \ \forall h\in \mathcal{C}^0_b$:

Osserviamo che $h\circ f\in \mathcal{C}^0_b$:

siccome per TLC: $\frac{S_n}{\sqrt{n}} \convlegge Z$, si ha la Tesi 1.

\subsection{Tesi 2: $\left(Y_n\right)_{n\in \mathbb{N}}$ UI}
\[sup_n \|Y_n\| < \infty \]
In quanto sono tutte funzioni $\mathbb{L}^1$ (infatti asintoticamente $f$ cresce al massimo come $x^4$, e $X_n\in \mathbb{L}^4$) e tendono in legge ad una funzione $\mathbb{L}^1$ (sempre per lo stesso motivo).

Inoltre $\mathbb{E}\left[|Y_n|; |Y_n|>L\right] \rightarrow  \mathbb{E}\left[|Y|;|Y|>L\right] < \varepsilon $ (sempre per convergenza in legge). 


\section{Esercizio 5.5}
Sappiamo che $\frac{S_n}{n} \longrightarrow \mu $ q.c. (per LGN; $\mu := \mathbb{E}[X_n]$).









% Per il teorema di Ergorov (è quasi vero che una funzione convergente puntualmente converge anche uniformemente) si ha che 
% \[\mathbb{P}\left(\left|\frac{S_n}{n}\right| > \mu +42\right) < \varepsilon  \]



\end{document}
