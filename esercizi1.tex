\documentclass{article}
\usepackage{graphicx} % Required for inserting images
\usepackage{amsmath}
\usepackage{listings}
\usepackage{textcomp}
\usepackage{multirow}
\usepackage{multicol}
\usepackage{booktabs}
\usepackage{graphicx}
\usepackage{floatflt}
\usepackage{epsfig}
\usepackage{pstricks}
\usepackage{subfigure}
\usepackage[labelfont=bf, font=scriptsize]{caption}
\usepackage[italian]{varioref}
\usepackage[suftesi]{frontespizio}
\usepackage{color}
\usepackage{tikz}
\usepackage{caption}
\usepackage{pgfplots}
\usepackage{comment}
\usepackage{lipsum}
\usepackage{hyperref}
\pgfplotsset{compat=1.16}
\usepackage[table]{xcolor}% http://ctan.org/pkg/xcolor
\usepackage{phoenician}
\usepackage{dsfont}

\usepackage{subfigure}
\usepackage[export]{adjustbox} % per l'allineamento delle immagini
\usepackage{float}
\usepackage{amssymb} % for the \lesssim symbol
\usepackage{makecell}
\usepackage{changepage}
\usepackage{amssymb}

\title{Esercizi prob 1}
\begin{document}
\maketitle
\section{Esercizio 1.1}
\subsection{(1)}
$F: \mathbb{R}\rightarrow \left[0,1\right]$ Ovvio

Non decrescenza: assumendo $x\leq y$:
\[
	F(y)= \mathbb{P}\left(X\leq y\right) = \mathbb{P}\left(X\leq x\right)
	+ \mathbb{P}
	\left(x < X \leq y \right)
	\geq
	\mathbb{P}\left(X\leq x\right) = F(x)
\]

\subsection{(2)}
Per convergenza monotona di Beppo Levi:
\[
	0 = \mathbb{P}\left(\bigcap_x \left\{X\leq x\right\}\right) = \lim_{y\rightarrow -\infty} F(x)
\]
Analogamente anche con $F(+\infty)$.

\subsection{(3)}
Sempre per convergenza monotona:
\[
	\lim_{y\rightarrow x^+} F(y) = \lim_{y\rightarrow x^+}\mathbb{P}\left(X\leq y\right) = \mathbb{P} \left(\bigcap_{y>x}\left\{X\leq y\right\}\right)
\]
\[
	F(x) = \mathbb{P}\left(X\leq x\right)	
\]
\subsection{Bonus}
\[
	\mathbb{P}\left(X=x\right) = \mathbb{P}\left(X\leq x\right) - \mathbb{P}\left(X<x\right) = F(x) - \mathbb{P}\left(\bigcup_{y<x} \left\{X\leq y\right\}\right) = F(x) - \lim_{y \nearrow x} F(y)
\]


\section{Esercizio 1.2}
\subsection{(1)}
\subsubsection{$X^+ \geq X^-$}
\[
	\left\{z| F(z)>\omega \right\} \subset \left\{z| F(z) \geq \omega\right\} \longrightarrow \inf \left\{z| F(z)>\omega \right\} \geq \inf \left\{z| F(z) \geq \omega\right\} 
\]

\subsubsection{$X^+(\omega) \in \mathbb{R} \ \forall \omega\in \left(0,1\right)$}
Supponiamo per assurdo che $X^+ (\omega) = +\infty$:
\[
	\forall z_0, \ \exists z>z_0, \ F(z) \leq \omega < 1
\]
\[
	\lim_{z\rightarrow +\infty} F(z) \leq \omega < 1	
\]
Che è in contraddizione con ipotesi (2).
\subsubsection{$X^-(\omega) \in \mathbb{R} \ \forall \omega\in \left(0,1\right)$}
Analogo a precedente.


\subsection{(2)}
\[
	X^-(\omega) \leq y \  \Longleftrightarrow  \  \inf \left\{z| F(z) \geq \omega\right\} \leq y \  \Longleftrightarrow  \  F(y) \geq \omega
\]
Ultimo passaggio vale perché $F(\cdot)$ è debolmente crescente.

\subsection{(3)}
Diamo per buoni tutti i suggerimenti.

Tesi: $\exists! \omega \ t.c. \ X^-(\omega) < q < X^+(\omega)$

Abbiamo che, $\forall \varepsilon > 0$:
\[
	X^-(\omega+\varepsilon) = \inf \left\{z| F(z) \geq \omega+\varepsilon\right\} \geq \inf \left\{z| F(z) > \omega\right\} > q
\]
Quindi, dato un $\omega$ che soddisfa la tesi, nessun $\omega$ strettamente maggiore di esso la soddisfa. Analogamente, nessun $\omega$ inferiore di esso la soddisfa.

\section{Esercizio 1.3}
\subsection{(1): $\mathrm{Exp}(\lambda)$}
\[
	F_X(x)= \int_{-\infty }^{x}f_X(x)\,dx = \int_{-\infty }^{x} \lambda e^{-\lambda x} \mathds{1}_{(0,\infty )}(x) \,dx = 1-e^{-\lambda x}
\]
\[
	X^+(\omega) = \inf\left\{x| F_X(x)>\omega\right\} = \inf\left\{x| 1-e^{-\lambda x} > \omega\right\} = 
\]
\[	= \inf\left\{x>0| x>- \frac{\ln\left(1 - \omega\right)}{\lambda} \right\} = - \frac{\ln\left(1 - \omega\right)}{\lambda}\]

\subsection{(2): $\mathrm{Cauchy}(\gamma)$}
\[
	F_X(x) = \frac{\arctan \frac{x}{\gamma}}{\pi} + \frac{1}{2}
\]
\[
	X^\pm(\omega) = \gamma \tan \left[\pi\left(\omega - \frac{1}{2}\right)\right]
\]


\subsection{(3): $P_X = pP_Y + (1-p)P_0$}
\[
	P_X(x) = p\left(e^{-x} \mathds{1}_{(0,\infty )}(x)\right) + (1-p) \delta(x)
\]
\[
	F_X(x) = \int_{-\infty }^{x} P_X \left(\xi\right) \, d\xi = \left[p\left(1-e^{-x}\right) + \left(1-p\right)\right] \mathds{1}_{[0,\infty )} (x)
\]
\[
	X^\pm(\omega) = -\ln \left(1- \frac{\omega-(1-p)}{p}\right) \mathds{1}_{[1-p,p]}(\omega)
\]

\subsection{(4): Valori $a_1, \dotsc, a_m$ con probabilità $p_1, \dotsc, p_m$}
\[
	f_X(x) = p_1 \delta(a_1) + \dotsb + p_m \delta(a_n)
\]
\[
	F_X(x) = \int_{-\infty }^{x} f_X \left(\xi\right) \, d\xi = s_{j(x)}
\]
Con $j(x) := \max\left\{j | a_j\leq x\right\}$

\[
	X^\pm (\omega) = \inf\left\{\omega | s_{j(x)} > \omega\right\}
\]
(Non sono riuscito a trovare una espressione migliore)

\subsection{(5): $F(x) = \left(1- pe^{-x}\right) \mathds{1}_{[0, \infty )}(x)$}
\[
	X^\pm\left(\omega\right) = - \ln\left(\frac{1-\omega}{p}\right) \mathds{1}_{[1-p,\infty )}(\omega)
\]

\subsection{(6): $f(x) = \frac{1}{2x^2} \mathds{1}_{(-1,1)}(x)$}
Non ne ho idea, visto che diverge con $x=0$

\end{document}
