\documentclass{article}
\usepackage{graphicx} % Required for inserting images
\usepackage{amsmath}
\usepackage{listings}
\usepackage{textcomp}
\usepackage{multirow}
\usepackage{multicol}
\usepackage{booktabs}
\usepackage{graphicx}
\usepackage{floatflt}
\usepackage{epsfig}
\usepackage{pstricks}
\usepackage{subfigure}
\usepackage[labelfont=bf, font=scriptsize]{caption}
\usepackage[italian]{varioref}
\usepackage[suftesi]{frontespizio}
\usepackage{color}
\usepackage{tikz}
\usepackage{caption}
\usepackage{pgfplots}
\usepackage{comment}
\usepackage{lipsum}
\usepackage{hyperref}
\pgfplotsset{compat=1.16}
\usepackage[table]{xcolor}% http://ctan.org/pkg/xcolor
\usepackage{phoenician}
\usepackage{dsfont}

\usepackage{subfigure}
\usepackage[export]{adjustbox} % per l'allineamento delle immagini
\usepackage{float}
\usepackage{amssymb} % for the \lesssim symbol
\usepackage{makecell}
\usepackage{changepage}
\usepackage{amssymb}
\newcommand{\myrightarrow}[1]{\xrightarrow{\makebox[2em][c]{$\scriptstyle#1$}}}
\newcommand{\convlegge}{\myrightarrow{\mathcal{L}}}

\newcommand{\ind}{\mathrel{\perp\!\!\!\perp}}
\title{Esercizi prob 3}
\begin{document}
\maketitle
\section{Esercizio 3.1}
Per BC1, $\forall  \varepsilon $ 
\[
	\mathbb{P}\left(|X_n - X| >\varepsilon , i.o.\right) = 0
\]

Ovvero: 
\[
	\forall \varepsilon , \ per\ q.o.\ \omega ,\  \exists N\ :\ |X_n(\omega ) - X(\omega )| \leq \varepsilon 
\]
Che è quasi la tesi (basta prendere solo una quantità numerabile di $\varepsilon $ che però tendano a $0$). $\Omega _0$ sarà l'intersezione (numerabile) di tutti gli insiemi per cui vale la formula scritta sopra.

\section{Esercizio 3.2}
\subsection{(1)}
\[
	\mathbb{P}\left(\left\{X_n \neq  0\right\}\right) = \frac{1}{k(n)} \rightarrow 0
\]

\subsection{(2)}
Dato $\omega $, per ogni valore distinto di $k$ esiste un valore di $n$ tale che $X_n(\omega ) =1$, quindi $X_n\left(\omega \right) = 1$ infinite volte.

Quindi
\[
	\nexists\, \omega \ : \ X_n(\omega ) \rightarrow  0 
\]
(tranne forse $\omega =0$ o $\omega =1$ ma non ho voglia di controllare).


\section{Esercizio 3.3}
\subsection{$Y_n \rightarrow  0$ in $\mathbb{L}^p\ \ \ \forall p$}
\[
	\mathbb{E}\left[\left|Y_n\right|^p\right]=\int_{0}^{+\infty }\frac{|x|^p}{\log^p(n)}e^{-x}\,dx = \frac{c}{\log^p(n)} \myrightarrow{n\rightarrow \infty } 0
\]

\subsection{$\left(Y_n\right)$ non converge q.c.}
\[
	\mathbb{P}\left(Y_n>1\right) = \mathbb{P}\left(X_n > \log n\right) = e^{-\log n} = \frac{1}{n}
\]
\[
	\sum_n \mathbb{P}\left(Y_n>1\right) = +\infty 
\]

Quindi, per BC2,
\[
	Y_n(\omega )>1\ \ \ \ \ i.o.
\]


\section{Esercizio 3.4}
\[
	\mathbb{P}\left(|X_1|>x\right) = \left(\mathbb{P}\left(\bigcap_{n} \left\{|X_n| > x\right\}\right)\right)^{\frac{1}{n}} = \exp \left[\frac{1}{n} \ln \mathbb{P}\left(\bigcap_n \left\{|X_n| > x\right\}\right)\right]
\]

\section{Esercizio 3.5}
\subsection{(1): $\#\left(C_F^0\right)^c \leq  \aleph_0$  ($\Rightarrow \overline{C_F^0} = \mathbb{R}$)}
$F$ è continuo a destra: $\forall x\in (C_F^0)^c$ si possono definire $$\omega_x^- := \lim_{y \nearrow x} F(y); \ \ \ \ \ \omega _x^+ := F(x)$$.

Si ha che $\exists q_x \in  (\omega _x^-, \omega _x^+) \ t.c.\ q_x\in \mathbb{Q}$. Inoltre, siccome $F$ è non descrescente, $q_x \neq  q_y\ \  \forall x\neq y$, quindi abbiamo creato una funzione iniettiva $(C_F^0)^c \longrightarrow \mathbb{Q}$

\subsection{(2): Eventualmente $F_n(z)>\omega \ \land \ X_n^+(\omega )\leq z$}
\[
	z > \inf\left\{y | F_X(y) > \omega \right\} \ \ \ \Longrightarrow \ \ \ F_X(z) > \omega 
\]

Siccome per ipotesi $z \in  C_F^0$:
\[
	F_{X_n} \left(z\right) \longrightarrow  F_X(z) > \omega 
\]
Quindi la prima tesi eventualmente è vera.

Siccome $F_{X_n}$ è debolmente crescente, questo implica anche che
\[
	z \geq  \inf\left\{y | F_{X_n}(y) > \omega \right\}
\]

\subsection{(3): $\limsup_n X_n^+(\omega ) \leq  X^+(\omega )$}
\[
	\limsup_n X_n^+(\omega ) \leq z\ \ \ \ \ \ \forall  z> X^+(\omega )
\]

\subsection{(4): $\liminf_n X_n^-(\omega ) \geq  X^-(\omega )$}
Basta rifare tutto invertendo tutti i segni.

\subsection{(5): $X_n^+ \rightarrow  X^+ \ \ q.c.$}
Q.c., $X^- = X^+$, quindi convergenza c'è per teorema dei 2 carabinieri.


\section{Esercizio 3.6}
\subsection{(1): $X_n \convlegge X_1$}
Nota: non ho capito la vera definizione formale di IID

Supponiamo che $\mathbb{E}\left[h(X_n)\right] \longrightarrow \mathbb{E}\left[H(X)\right]$ con $h$ semplice. Allora sappiamo che tutte le funzioni continue bounded sono il limite di una serie crescente di funzioni semplici. Quindi tesi vera per convergenza dominata.


\subsection{(2): $\exists  a<b \ : \ \mathbb{P}\left(X_1\leq a\right) > 0 \ \land  \ \mathbb{P}\left(X_1\geq b\right) > 0$}
Assumendo la tesi falsa per assurdo, siccome
\[
	\mathbb{P}\left(X_1\leq a\right) + \mathbb{P}\left(a < X_1 < b\right) + \mathbb{P}\left(b\leq X_1\right) = 1
\]

Si ha che $\mathbb{P}\left(a < X_1 < b\right) = 1\ \ \ \forall\, a<b$.

Quindi $X_1$ è costante (che va contro l'ipotesi).

La seconda tesi è vera per BC2.

\subsection{(3): $\left(X_n\right)_n$ non converge q.c.}
Q.c. $\left(X_n (\omega )\right)_n$ continua a saltare da sotto $a$ a sopra $b$ infinite volte, quindi non può convergere.

\subsection{(4): $\left(X_n\right)_n$ non converge in prob}
Per qualsiasi $X(\omega )$ (ad $\omega $ fissato) si ha che 
\[
	\left|X_n(\omega ) - X\left(\omega \right)\right| > \left(b-a\right)
\]
infinite volte.


\end{document}
