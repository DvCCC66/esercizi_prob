\documentclass{article}
\usepackage{graphicx} % Required for inserting images
\usepackage{amsmath}
\usepackage{listings}
\usepackage{textcomp}
\usepackage{multirow}
\usepackage{multicol}
\usepackage{booktabs}
\usepackage{graphicx}
\usepackage{floatflt}
\usepackage{epsfig}
\usepackage{pstricks}
\usepackage{subfigure}
\usepackage[labelfont=bf, font=scriptsize]{caption}
\usepackage[italian]{varioref}
\usepackage[suftesi]{frontespizio}
\usepackage{color}
\usepackage{tikz}
\usepackage{caption}
\usepackage{pgfplots}
\usepackage{comment}
\usepackage{lipsum}
\usepackage{hyperref}
\pgfplotsset{compat=1.16}
\usepackage[table]{xcolor}% http://ctan.org/pkg/xcolor
\usepackage{phoenician}
\usepackage{dsfont}

\usepackage{subfigure}
\usepackage[export]{adjustbox} % per l'allineamento delle immagini
\usepackage{float}
\usepackage{amssymb} % for the \lesssim symbol
\usepackage{makecell}
\usepackage{changepage}
\usepackage{amssymb}

\newcommand{\ind}{\mathrel{\perp\!\!\!\perp}}
\title{Esercizi prob 3}
\begin{document}
\maketitle
\section{Esercizio 3.1}
Per BC1, $\forall  \varepsilon $ 
\[
	\mathbb{P}\left(|X_n - X| >\varepsilon , i.o.\right) = 0
\]

Ovvero: 
\[
	\forall \varepsilon , \ per\ q.o.\ \omega ,\  \exists N\ :\ |X_n(\omega ) - X(\omega )| \leq \varepsilon 
\]
Che è quasi la tesi (basta prendere solo una quantità numerabile di $\varepsilon $ che però tendano a $0$). $\Omega _0$ sarà l'intersezione (numerabile) di tutti gli insiemi per cui vale la formula scritta sopra.

\section{Esercizio 3.2}
\subsection{(1)}
\[
	\mathbb{P}\left(\left\{X_n \neq  0\right\}\right) = \frac{1}{k(n)} \rightarrow 0
\]

\subsection{(2)}
Dato $\omega $, per ogni valore distinto di $k$ esiste un valore di $n$ tale che $X_n(\omega ) =1$, quindi $X_n\left(\omega \right) = 1$ infinite volte.

Quindi
\[
	\nexists\, \omega \ : \ X_n(\omega ) \rightarrow  \omega 
\]
(tranne forse $\omega =0$ o $\omega =1$ ma non ho voglia di controllare).


\section{Esercizio 3.3}
\subsection{$Y_n \rightarrow  0$}

\subsection{$\left(Y_n\right)$ non converge q.c.}


\section{Esercizio 3.4}
\[
	\mathbb{P}\left(|X_1|>x\right) = \left(\mathbb{P}\left(\bigcap_{n} \left\{|X_n| > x\right\}\right)\right)^{\frac{1}{n}} = \exp \left[\frac{1}{n} \ln \mathbb{P}\left(\bigcap_n \left\{|X_n| > x\right\}\right)\right]
\]

\end{document}
